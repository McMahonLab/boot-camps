\documentclass[12pt,t]{beamer}
\usepackage{graphicx}
\setbeameroption{hide notes}

%%%%%%%%%%%%%%%%%%%%%%%%%%%%%%%%%%%%%%%%%%%%%%%%%%%%%%%%%%%%%%%%%%%%%%
% header stuff
%%%%%%%%%%%%%%%%%%%%%%%%%%%%%%%%%%%%%%%%%%%%%%%%%%%%%%%%%%%%%%%%%%%%%%

% get rid of junk
\usetheme{default}
\beamertemplatenavigationsymbolsempty
\hypersetup{pdfpagemode=UseNone} % don't show bookmarks on initial view

% font
\usepackage{fontspec}
\setsansfont
  [ ExternalLocation = fonts/ ,
    UprightFont = *-regular ,
    BoldFont = *-bold ,
    ItalicFont = *-italic ,
    BoldItalicFont = *-bolditalic ]{texgyreheros}
\setbeamerfont{note page}{family*=pplx,size=\footnotesize} % Palatino for notes
% "TeX Gyre Heros can be used as a replacement for Helvetica"
% I've placed them in ../fonts/; alternatively you can install them
% permanently on your system as follows:
%     Download http://www.gust.org.pl/projects/e-foundry/tex-gyre/heros/qhv2.004otf.zip
%     In Unix, unzip it into ~/.fonts
%     In Mac, unzip it, double-click the .otf files, and install using "FontBook"

% named colors
\definecolor{offwhite}{RGB}{255,250,240}
\definecolor{gray}{RGB}{155,155,155}

\definecolor{background}{RGB}{255,255,255}
\definecolor{foreground}{RGB}{24,24,24}
\definecolor{title}{RGB}{27,94,134}
\definecolor{subtitle}{RGB}{22,175,124}
\definecolor{hilit}{RGB}{122,0,128}
\definecolor{vhilit}{RGB}{255,0,128}
\definecolor{lolit}{RGB}{95,95,95}

\newcommand{\hilit}{\color{hilit}}
\newcommand{\vhilit}{\color{vhilit}}
\newcommand{\nhilit}{\color{nhilit}}
\newcommand{\nvhilit}{\color{nvhilit}}
\newcommand{\lolit}{\color{lolit}}

% use those colors
\setbeamercolor{titlelike}{fg=title}
\setbeamercolor{subtitle}{fg=subtitle}
\setbeamercolor{institute}{fg=lolit}
\setbeamercolor{normal text}{fg=foreground,bg=background}
\setbeamercolor{item}{fg=foreground} % color of bullets
\setbeamercolor{subitem}{fg=lolit}
\setbeamercolor{itemize/enumerate subbody}{fg=lolit}
\setbeamertemplate{itemize subitem}{{\textendash}}
\setbeamerfont{itemize/enumerate subbody}{size=\footnotesize}
\setbeamerfont{itemize/enumerate subitem}{size=\footnotesize}

% page number
\setbeamertemplate{footline}{%
    \raisebox{5pt}{\makebox[\paperwidth]{\hfill\makebox[20pt]{\lolit
          \scriptsize\insertframenumber}}}\hspace*{5pt}}

% add a bit of space at the top of the notes page
\addtobeamertemplate{note page}{\setlength{\parskip}{12pt}}

% default link color
\hypersetup{colorlinks, urlcolor={hilit}}

% a few macros
\newcommand{\bi}{\begin{itemize}}
\newcommand{\bbi}{\vspace{24pt} \begin{itemize} \itemsep8pt}
\newcommand{\ei}{\end{itemize}}
\newcommand{\ig}{\includegraphics}
\newcommand{\subt}[1]{{\footnotesize \color{subtitle} {#1}}}
\newcommand{\ttsm}{\tt \small}
\newcommand{\ttfn}{\tt \footnotesize}
\newcommand{\figh}[2]{\centerline{\includegraphics[height=#2\textheight]{#1}}}
\newcommand{\figw}[2]{\centerline{\includegraphics[width=#2\textwidth]{#1}}}


%%%%%%%%%%%%%%%%%%%%%%%%%%%%%%%%%%%%%%%%%%%%%%%%%%%%%%%%%%%%%%%%%%%%%%
% end of header
%%%%%%%%%%%%%%%%%%%%%%%%%%%%%%%%%%%%%%%%%%%%%%%%%%%%%%%%%%%%%%%%%%%%%%

\title{Make Incremental Changes}
\subtitle{Version Control with git and GitHub}
\date{}

\begin{document}

{
\setbeamertemplate{footline}{} % no page number here
\frame{ \titlepage }
}

\begin{frame}[c]{}

% comic from http://www.phdcomics.com/comics/archive.php?comicid=1531
\figh{pics/phd101212s.png}{0.9}

\vfill
\color{lolit} \tiny
\centerline{\url{http://www.phdcomics.com/comics/archive.php?comicid=1531}}

\end{frame}




\begin{frame}{\only<1>{Methods for tracking versions}\only<2>{Suppose it stops working\dots}}
\bbi
\item Don't keep track
\onslide<2>{
\bi
\item Good luck!
\ei
}
\item Save numbered files
\onslide<2>{
\bi
\item Manually compare files (with {\tt diff})
\ei
}
\item Formal version control
\onslide<2>{
\bi
\item Easy to study changes back in time
\item Easy to jump back and test
\ei
}
\ei

\end{frame}




\begin{frame}[c]{}

\vspace*{2cm}

\figw{pics/kcranstn_tweet.png}{0.7}

\vspace*{3cm}

\color{lolit} \tiny
\centerline{\url{http://bit.ly/kcranstn_tweet}}

\end{frame}



\begin{frame}{Why use formal version control?}
\bbi
\item Backup files that are changing
\item Store a history of those changes
\item Explore that history
\item No worries about breaking things that work
\item Merge changes from multiple people
\ei

\end{frame}





\begin{frame}[c]{Example repository}

\figh{pics/MDTPsstevens_repo.png}{0.9}

\end{frame}


\begin{frame}[c]{Example history}

\figh{pics/MDTPsstevens_commits.png}{0.9}

\end{frame}




\begin{frame}[c]{Example commit}

\figh{pics/MDTPsstevens_acommit.png}{0.9}
\end{frame}


\begin{frame}{Why git?}
\bbi
\item It's fast
\item You don't need access to a server
\item Amazingly good at merging simultaneous changes
\item It's popular
\item GitHub
\onslide<2>{
  \bi
  \item A home for git repositories
  \item Interface for exploring git repositories
  \item Like facebook for programmers (and data scientists)
  \ei
}
\ei
\end{frame}


\begin{frame}{It's not just for software}

\vspace*{5mm}
You can track any content {\lolit (but mostly plain text files)}
\vspace*{-5mm}

\bbi
\item source code
\item data analysis projects
\item manuscripts
\item presentations
\item websites
\ei

\end{frame}


\end{document}
